%%%%%%%%%%%%%%%%%%%%%%%%%%%%%%%%%%%
%
%
%  Science Fair Project 
%  
%  by Theodore Ehrenborg
%
%
%
%%%%%%%%%%%%%%%%%%%%%%%%%%%%%%%%%
%
% Last edited: February 28, 2019
%
%%%%%%%%%%%%%%%%%%%%%%%%%%%%%%%%%
%%%%%%%%%%%%%%%%%%%%%%%%%%%%%%%%%
%
% pdf settings
%
%%%%%%%%%%%%%%%%%%%%%%%%%%%%%%%%%
%
%
%\pdfpagewidth=8.5truein
%\pdfpageheight=11truein


%
%%%%%%%%%%%%%%%%%%%%%%%%%%%%%%%%%



\documentclass[12pt]{article}
\usepackage[title]{appendix}
\usepackage[font=Large,labelfont=bf]{caption}
%\usepackage{abstract}
\usepackage{anyfontsize}
\usepackage{dirtytalk}
\usepackage{tikz}
\usepackage{amssymb, amsmath, fullpage, amsthm}
\usepackage{mathrsfs}
\usepackage{ gensymb }
\usepackage{pdflscape}%Prepares to flip the document
\usepackage{todonotes}
\usepackage{enumerate}


\usetikzlibrary{math}

\parskip3mm

\newtheorem{theorem}{Theorem}
\newtheorem{property}[theorem]{Property}
\newtheorem{hypothesis}[theorem]{Hypothesis}
\newtheorem{lemma}[theorem]{Lemma}
\newtheorem{parametrization}[theorem]{Parametrization}
\newtheorem{proposition}[theorem]{Proposition}
\newtheorem{definition}[theorem]{Definition}
\newtheorem{corollary}[theorem]{Corollary}
\newtheorem{example}[theorem]{Example}
\newtheorem{remarks}[theorem]{Remarks}
\newtheorem{remark}[theorem]{Remark}

\numberwithin{equation}{section}
\usepackage[bindingoffset=0.2in,
            left=0.25in,right=0.5in,top=0.25in,bottom=0.25in,
            footskip=.25in]{geometry}%Sets margins
\pagenumbering{gobble}%No page numbers

\newcommand{\controlledbf}{}
\newcommand{\MySpacing}{25}
%\newcommand{\MySpacing}{10}
\newcommand{\MyFontSize}{20}
%\newcommand{\MyFontSize}{10}
\newcommand{\MySectionSpacing}{15}
%\newcommand{\MySectionSpacing}{10}
\newcommand{\MySectionFontSize}{40}
%\newcommand{\MySectionFontSize}{10}
\newcommand{\timesdots}{\times \cdots \times}
\newcommand{\vanish}[1]{}
\newcommand{\coveredby}{\prec}

%\renewcommand{\abstractnamefont}{\fontsize{\MyFontSize}{\MySpacing}}
%\renewcommand{\abstracttextfont}{\normalfont\small}


\DeclareMathOperator{\inv}{inv}
\DeclareMathOperator{\frst}{frst}
\DeclareMathOperator{\er}{er}
\DeclareMathOperator{\asc}{asc}
\DeclareMathOperator{\odd}{odd}

\newcommand{\ascodd}{\asc_{\odd}}

\newcommand{\doubleprime}{\prime\prime}


\font\Cp = msbm10

\newcommand{\Ccc}{\hbox{\Cp C}}
\newcommand{\Fff}{\hbox{\Cp F}}
\newcommand{\Hhh}{\hbox{\Cp H}}
\newcommand{\Nnn}{\hbox{\Cp N}}
\newcommand{\Rrr}{\hbox{\Cp R}}
\newcommand{\Sss}{\hbox{\Cp S}}
\newcommand{\Zzz}{\hbox{\Cp Z}}
%The above commands don't work with large font.

\newcommand{\SSSS}{\mathfrak{S}}

\newcommand{\Gaussian}[2]{\genfrac{[}{]}{0pt}{}{#1}{#2}_q}



\newcommand{\divides}{\mid}
\newcommand{\notdivides}{\nmid}

\newcommand{\fix}[1]{\todo[inline]{#1}}

\begin{document}

\begin{landscape}

\bf
{
\title
{
\fontsize{120}{150}\selectfont%1st number is font, 2nd is spacing 
\begin{center}
\textbf { Classifying Quaternion Identities }
\end{center}
}
}
\author{
\fontsize{50}{50}\selectfont
\sc Theodore Ehrenborg}
\date{}
\maketitle

\newpage

{\fontsize{\MyFontSize}{\MySpacing}\selectfont%1st number is font, 2nd is spacing 
{


\noindent
\begin{center}
{\fontsize{\MySectionFontSize}{\MySectionSpacing}\selectfont \textbf
 {Abstract}
}
\end{center}

\fontsize{\MyFontSize}{\MySpacing}\selectfont

This number theory project investigates identities found by
multiplying together quaternions in \( \mathbb{L}[x,y,z,w] \), the
Lipschitz quaternions \( \mathbb{L} \) adjoined with the
indeterminates \(x\), \(y\), \(z\), \(w\).  Recall that quaternions
are \(4\)-dimensional complex numbers.  These identities provide
solutions to \( \sum_{j = 1}^{p} \tau_j ^ 2 = \left( \sum_{i = 1}^{m}
x_i ^ 2 \right) ^ n \). We present a rigorous definition that captures
the intuitive notion of when two such identities are equivalent. This
definition implies that the true structure of this problem involves the
group action of the direct product \( \mathfrak{S}_4^\pm \times \mathfrak{S}_4^\pm \).  Using
two complementary methods, we compute the number of equivalence
classes for \(n = 1, 2, 3, 4,\) where \(n\) is the number of
quaternion factors. We move to the case concerning products of complex
numbers, namely \( \mathbb{Z}[i][x,y] \). Using the fact that the
Gaussian integers are commutative under multiplication, we
characterize these equivalence classes, thus also providing an
enumeration.

\newpage

\noindent
\begin{center}
{\fontsize{\MySectionFontSize}{\MySectionSpacing}\selectfont \textbf
 {Introduction}
}
\end{center}


The following group is an essential part of Definition~\ref{def:general}
and Definition~\ref{def:2D}.
\begin{definition}
The {\bf symmetric group} \( \mathfrak{S}_n \) is the 
set of all permutations \( \pi = \pi_1 \cdots \pi_n \) 
of the \( n \) element set \( \{ 1, 2, \ldots, n \} \),
where \( \pi(i) = \pi_i \).
The {\bf signed symmetric group} \( \mathfrak{S}_n^\pm \)
is the set of all permutations \( \sigma = \sigma_1 \cdots \sigma_n\)
of the set \( \{ \pm 1, \pm 2, \ldots, \pm n \} \) such that
\( | \sigma | = | \sigma_1 | \cdots |\sigma_n| \in \mathfrak{S}_n \).
\end{definition}

For a signed permutation \( \pi \in \mathfrak{S}_m^\pm \), let \( \pi \) act on a polynomial in the 
\(m\) variables \( x_1,x_2, \ldots, x_m \) by sending \( x_j \) to 
\[
\pi(x_j) =
\begin{cases}
x_{\pi_j} & \text{if } \pi_j > 0 \\
-x_{-\pi_j} & \text{if } \pi_j < 0
\end{cases}
\]

\begin{definition}
\label{def:general}
Fix \( p, m \in \mathbb{N} \). 
Let \( \tau = ( \tau_1, \ldots, \tau_p) \)
be a tuple of length \( p \) where 
\( \tau_i \in \mathbb{Z}[x_1,x_2, \ldots, x_m] \), where \( i = 1, \ldots, p \).
We define an equivalence relation, denoted by \( \simeq \), on \(p\)-tuples
\( \tau \) 
by taking the transitive closure of the following three relations:
\begin{itemize}
\item
\( ( \tau_1, \ldots, \tau_p) \simeq ( \tau'_1, \ldots, \tau'_p) \)
if there exists a signed permutation \( \pi \in \mathfrak{S}_m^\pm \)
acting on the \( m \) variables such that \( \pi( \tau_i ) = \tau'_i \),
 where \( i = 1, \ldots, p \).
\item
\( ( \tau_1, \ldots, \tau_p) \simeq ( \tau'_1, \ldots, \tau'_p) \)
if there exists a permutation \( \sigma \in \mathfrak{S}_p \)
such that \( \tau_{\sigma(i)} = \tau'_i \), where \( i = 1, \ldots, p \).
\item
\( ( \tau_1, \ldots, \tau_p) \simeq ( \pm \tau_1, \ldots, \pm \tau_p) \)
\end{itemize}
\end{definition}


\begin{example}
\bf
According to Definition~\ref{def:general}, the following is true:
\begin{align*}
( xz,\: y^2,\: yz )  
&\simeq ( (-y)(-x),\: z^2,\: z(-x) ) \\
&\simeq ( (-y)(-x),\: z(-x),\: z^2 ) \\
&\simeq ( -(-y)(-x),\: z(-x),\: -z^2 ) 
\end{align*}

\end{example}



%Fix \( p \in \mathbb{Z} \). 
We are interested in counting the number of
equivalence classes of the set of all tuples \( \tau \) where 
%the sum of the squares of the elements of \( \tau \) is
\[
\sum_{j = 1}^{p}  \tau_j ^ 2  
= 
\left( \sum_{i = 1}^{m}  x_i ^ 2  \right) ^ n 
\] 
This problem is most easily attacked when we
view \( \left( \sum_{i = 1}^{m}  x_i ^ 2  \right) ^ n \)
as the norm of a product of complex numbers or quaternions.
Thus we will focus on the cases where \( p = m = 2 \) and where \( p = m = 4 \).
The general problem can also be viewed as finding the disjoint orbits of 
various tuples, where the group action is \( \mathfrak{S}_p^\pm \times \mathfrak{S}_m^\pm \).



\begin{example}
\bf
Consider the case where \( p = m = 2 \) and \( n = 2\).
The following two identities are representatives from 
the two different equivalence classes in this case. These identities
were generated by a product of complex numbers. 

\noindent
Identity i:
\begin{equation*}
(x + iy)(x + iy) = (x^2 - y^2 ) + i(2xy) 
\end{equation*}
Taking norms, this gives the identity:
\begin{equation*}
    (x^2 - y^2 )^2 + (2xy)^2 
    = (x^2 + y^2)^2
\end{equation*}
Identity ii:
\begin{equation*}
    (x + iy )(x - iy )
    = (x^2 + y^2 ) + i(0)  
\end{equation*}
Taking norms, this gives the identity:
\begin{equation*}
    (x^2 + y^2 )^2 + (0)^2
    = (x^2 + y^2 )^2
\end{equation*}
\end{example}



\begin{example}
\bf
Consider the case where \( p = m = 2 \) and \( n = 3\).
The following two identities are representatives from 
the two different equivalence classes in this case. These identities
were generated by a product of complex numbers. 

\noindent
Identity iii:
\begin{align*}
    (x + iy)(x + iy)(x + iy) 
    = x(x^2 - 3y^2) + i(  y(3x^2 - y^2) )  
    \end{align*}
Taking norms, this gives the identity:
    \begin{align*}
    (x(x^2 - 3y^2))^2 + (  y(3x^2 - y^2) )^2  
    = (x^2 + y^2)^3
    \end{align*}
Identity iv:
    \begin{align*}
    (x + iy )(x + iy)(x - iy ) 
    = x(x^2 + y^2 ) + i(y(x^2 + y^2))  
    \end{align*}
Taking norms, this gives the identity:
    \begin{align*}
    ( x(x^2 + y^2) )^2 + ( y(x^2 + y^2) )^2 
    = (x^2 + y^2 )^3
    \end{align*}
\end{example}














\newpage

\noindent
\begin{center}
{\fontsize{\MySectionFontSize}{\MySectionSpacing}\selectfont \textbf
 {The case where \( p = m = 2\)}
}
\end{center}

In the case where \( p = m = 2\), Definition~\ref{def:general} has an alternate form.



\begin{definition}
\label{def:2D}

Let \( h(z), h'(z) \in \mathbb{Z} [i][x,y] \),
where \( z = x + iy \).
Let \( M \) be the following set of mappings: 
\[
M = \{ z \mapsto uz \mid u \in \{ \pm 1, \pm i \} \}  
\cup \{ z\mapsto u \bar{z} \mid u \in \{ \pm 1, \pm i \} \}  
\]
%Write \( f(x,y) + i \cdot g(x,y) \) and \( f'(x,y) + i \cdot g'(x,y) \) as  \( h(z) \) and  \( h'(z) \), respectively.
We say \( h(z) \simeq h'(z) \) when there exist mappings \( \varphi, \varphi' \in M \)
such that \( \varphi( h( \varphi'( z ) ) )  = h'(z) \).

%\todo{ Prove that this is an equivalence relation. 
%That is, it has the reflexive, symmetric, and transitive properties. }

\end{definition}

\begin{lemma}
\bf
The relation \( \simeq \) is an equivalence relation.
\end{lemma}

\begin{proof}
\bf
The relation \( \simeq \) satisfies the three conditions of an equivalence relation.
\begin{enumerate}
\item Reflexive Property: If we choose \( \mu \) and \( \mu'\) to be the identity map \( z \mapsto z \), 
then \( h(z) \simeq h(z) \).

\item Symmetric Property: Let  \( h(z) \simeq h'(z) \), that is, there exist mappings 
\( \mu, \mu' \in M \)
such that \( \mu( h( \mu'( z ) ) )  = h'(z) \).
\(M\) is isomorphic to the signed symmetric group \( \mathfrak{S}_2^\pm \), so every mapping in \(M\)
has an inverse in \(M\). As  \( \mu^{-1}( h'( (\mu')^{-1}( z ) ) )  = h(z) \),
we have \( h'(z) \simeq h(z) \).

\item Transitive Property: Let \( a(z) \simeq b(z) \) and \( b(z) \simeq c(z) \). By the 
Symmetric Property, we have \( c(z) \simeq b(z) \). Thus there exist mappings
\( \mu, \mu', \nu, \nu' \in M \) such that \( \mu( a( \mu'( z ) ) )  = b(z) \) and 
\( \nu( c( \nu'( z ) ) )  = b(z) \). As a result:
\[
\mu( a( \mu'( z ) ) ) = \nu( c( \nu'( z ) ) )  
\]
Thus we have: 
\[
a(  z  ) = \mu^{-1}( \nu( c( (\mu')^{-1}( \nu'( z ) ) ) ) )
\]
This means that \( c(z) \simeq a(z) \) or \( a(z) \simeq c(z) \).
\end{enumerate}
\end{proof}
Recall that given \( f(x,y), g(x,y) \in \mathbb{Z}[x,y] \),
we can find \( h(z) \in \mathbb{Z} [i][x,y] \)
such that
\( z = x + iy \)
and
\( h(z) = f(x,y) + i \cdot g(x,y) \). 
The converse is also true.



\begin{lemma}
\bf Let \( h(z) \simeq h'(z) \), where \( h(z), h'(z) \in \mathbb{Z}[i][x,y] \)
 and \( z = x+ iy \). Suppose \( ( x^2 + y^2 ) ^ u
    \divides h(z) \), where \( u \in \mathbb{N} \). Then \( ( x^2 + y^2 ) ^ u \divides h'(z) \).
\end{lemma}
   
\begin{proof}
\bf
Let \( h(z) = ( x^2 + y^2 ) ^ u  p(z) \), where \( p(z) \in \mathbb{Z} [i][x,y] \). 
Whatever mappings we apply to \(h\) and \(z\)
to get \( h'(z) \), we also apply to the factors of \( h(z) \). 
No mapping will remove the factor of \( ( x^2 + y^2 ) ^ u \).

\end{proof}

\begin{corollary}
\bf
Let \( h(z), h'(z) \in \mathbb{Z}[i][x,y] \) with \( z = x+ iy \).
\[
h(z) \simeq h'(z) 
\implies 
( \forall u \in \mathbb{N} \cup \{ 0 \},  ( x^2 + y^2 ) ^ u \divides h(z)  \iff ( x^2 + y^2 ) ^ u \divides h'(z)  )
\]
\end{corollary}




\begin{theorem}
\bf
Consider the set of all \(2\)-tuples \( \tau \) where 
\(
  \tau_1 ^ 2   +   \tau_2 ^ 2   
= 
\left(  x_1 ^ 2 + x_2 ^ 2  \right) ^ n 
\).
The number of equivalence classes within this set 
is exactly  \( \lfloor \frac{n}{2} \rfloor + 1 \). 
Moreover, each equivalence class contains a tuple
of the form \( ( \Re( \beta ) , \Im( \beta ) ) \),
where \( \beta = (x + iy)^j (x -  iy)^{n-j} \),
with \( j \)
being one of \( 0, 1, 2, \ldots, \lfloor \frac{n}{2} \rfloor \).
\end{theorem}



\begin{proof}
\bf
Suppose 
\[ (x^2 + y^2)^n = f(x,y)^2 + g(x,y)^2 \]
where \( f(x,y), g(x,y) \in \mathbb{Z}[x,y] \).
Then  
\[ (x + iy)^n (x -  iy)^n = ( f(x,y) + i \cdot g(x,y) ) ( f(x,y) - i \cdot g(x,y) ) .\]
Since \( \mathbb{Z} [i][x,y] \) is a unique factorization domain, 
\[ f(x,y) + i \cdot g(x,y) = (x + iy)^j (x -  iy)^k ( \pm 1 \text{ or} \pm i ) \]
and
\[ f(x,y) - i \cdot g(x,y) = (x + iy)^r (x -  iy)^s ( \pm 1 \text{ or} \mp i ), \]
where \( j, k, r, s \in \mathbb{N} \cup \{0\} \). 
We know \( j + r = n \) and \( k + s = n \), as well as (by taking norms) \( j + k = n \) and \( r + s = n \).

Thus \(r = k\) and \(s = j\).

Clearly \( f(x,y) + i \cdot g(x,y) \simeq  f(x,y) - i \cdot g(x,y) \).
Since 
\[ 
f(x,y) + i \cdot g(x,y) = (x + iy)^j (x -  iy)^{n-j} ( \pm 1 \text{ or} \pm i ) 
\]
and 
\[ 
f(x,y) - i \cdot g(x,y) = (x + iy)^{n-j} (x -  iy)^j ( \pm 1 \text{ or} \mp i ) ,
\]
each equivalence class contains a representative with \( j \leq n - j
\).  As \( 2j \leq n \), we have \( j \leq \frac{n}{2} \), so \( j \)
is one of \( 0, 1, 2, \ldots, \lfloor \frac{n}{2} \rfloor \). This shows
that there are at most \( \lfloor \frac{n}{2} \rfloor + 1 \)
equivalence classes. Now we show that there are at least that many.



Consider \( (x + iy)^v (x -  iy)^{n-v} \)  and \( (x + iy)^u (x -  iy)^{n-u} \),
where \( v \neq u \) and \(v,u \leq [ \frac{n}{2} ] \). We have:
\begin{align*}
(x + iy)^v (x -  iy)^{n-v} &= (x^2 + y^2)^v (x -  iy)^{n-2v}
\\
(x + iy)^u (x -  iy)^{n-u} &= (x^2 + y^2)^u (x -  iy)^{n-2u}
\end{align*}

Without loss of generality, \( v > u \). 
Since \( \mathbb{Z} [i][x,y] \) is a unique factorization domain,
\( (x^2 + y^2) \notdivides (x -  iy)^m \) for \( m \in \mathbb{N} \cup \{ 0 \} \).
Thus \( (x^2 + y^2)^v \divides (x + iy)^v (x -  iy)^{n-v} \)
but \( (x^2 + y^2)^v \notdivides (x + iy)^u (x -  iy)^{n-u} \).

Therefore \( (x + iy)^v (x -  iy)^{n-v} \not\simeq (x + iy)^u (x -  iy)^{n-u} \), 
so the \( \lfloor \frac{n}{2} \rfloor + 1 \) representatives come from
distinct equivalence classes.

Thus there are \( \lfloor \frac{n}{2} \rfloor + 1 \) equivalence classes, 
one each for \( v = 0, 1, 2, \ldots,  \lfloor \frac{n}{2} \rfloor \) .
\end{proof}









\begin{figure}[h]
\label{fig:2D}

\begin{center}
\begin{tikzpicture}

\tikzmath{
\hash = 0.1;
\height = 6;
\length = 10;
\scale = 2;
}


\draw[thick] (0,0) -- (\scale * \length,0);
\fill (\scale * \length/2, -1) circle (0cm) node[anchor=north] {\Large \( n \) };

\draw[thick] (0,0) -- (0,\scale * \height);
%\fill (-2, \scale * \height/2 ) circle (0cm) node[rotate=90,anchor=south] {\Large  Number of };
%\fill (-1, \scale * \height/2 ) circle (0cm) node[rotate=90,anchor=south] {\Large  equivalence classes };
\fill (-1, \scale * \height/2 ) circle (0cm) node[anchor=east] {\Large  \( \kappa \) };


\foreach \y in {1, ..., \height}
    \draw[thick] ( -\hash, \scale * \y ) node[anchor=east] {\Large \y} -- ( \hash, \scale * \y );

\foreach \x in {1, ..., \length}
{
   \draw[thick] ( \scale * \x, -\hash ) node[anchor=north] {\Large \x} -- ( \scale * \x, \hash ) ;
   \fill (\scale * \x, { \scale *  floor( \x / 2 ) + \scale } ) circle (.1cm);
}
  

\end{tikzpicture}

\end{center}

\caption{
\bf The number of equivalence classes \( \kappa \)
when \(p = m = 2\) is \( \lfloor \frac{n}{2} \rfloor + 1 \) .
 }
\end{figure}





\begin{table}[p]

\label{table:4D}

\fontsize{\MySectionFontSize}{40}\selectfont%1st number is font, 2nd is spacing 
%\LARGE
\begin{center}


\begin{tabular}{ c | c }
 \( n \) & \( \kappa \) \\
\hline\hline
 1 & 1 \\
\hline
 2 & 8 \\
\hline
 3 & 48 \\
\hline
 4 & 965 
\end{tabular}




\end{center}

\caption{
\bf The conjectured number of equivalence
\\
classes \( \kappa \)  
in the case where \(p = m = 4\).
 }

\end{table}



\newpage

\noindent
\begin{center}
{\fontsize{\MySectionFontSize}{\MySectionSpacing}\selectfont \textbf
 {The case where \(p = m = 4\) and \(n = 2\)}
}
\end{center}








%Consider the following eight expressions, each of which is in
%\( \mathbb{Z}[i,j,k][x,y,z,w] \).
The following eight identities are representatives from 
eight different equivalence classes. These identities
were generated by a product of quaternions. 

\begin{enumerate}[{Identity} I:]
\item
    \begin{align*}
    &(x + iy + jz + kw)(x + iy + jz + kw) \\
    &= (x^2 - y^2 - z^2 - w^2 ) + i(2xy) + j(2xz) + k(2xw) 
    \end{align*}
Taking norms, this gives the identity:
    \begin{align*}
    &(x^2 - y^2 - z^2 - w^2 )^2 + (2xy)^2 + (2xz)^2 + (2xw)^2 \\
    &= (x^2 + y^2 + z^2 + w^2)^2
    \end{align*}
\item
    \begin{align*}
    &(x + iy + jz + kw)(x + iy + jz - kw) \\
    &= (x^2 - y^2 - z^2 + w^2 ) + i(2xy - 2zw) + j(2xz + 2yw) + k(0) 
    \end{align*}
Taking norms, this gives the identity:
    \begin{align*}
    &(x^2 - y^2 - z^2 + w^2 )^2 + (2xy - 2zw)^2 + (2xz + 2yw)^2 + (0)^2\\ 
    &= (x^2 + y^2 + z^2 + w^2)^2
    \end{align*}
\item
    \begin{align*}
    &(x + iy + jz + kw)(x - iy - jz - kw) \\
    &= (x^2 + y^2 + z^2 + w^2 ) + i(0) + j(0) + k(0) 
    \end{align*}
Taking norms, this gives the identity:
    \begin{align*}
    &(x^2 + y^2 + z^2 + w^2 )^2 + (0)^2 + (0)^2 + (0)^2 \\
    &= (x^2 + y^2 + z^2 + w^2)^2
    \end{align*}
\item
    \begin{align*}
    &(x + iy + jz + kw)(x + iy + jw + kz) \\
    &= (x^2 - y^2 - 2zw ) + i(2xy + z^2 -w^2) \\
        &+ j(xz - yz + xw + yw) + k(xz - yz + xw + yw) 
    \end{align*}
Taking norms, this gives the identity:
    \begin{align*}
    &(x^2 - y^2 - 2zw )^2 + (2xy + z^2 - w^2)^2 \\
        &+ (xz - yz + xw + yw)^2 + (xz - yz + xw + yw)^2 \\
    &= (x^2 + y^2 + z^2 + w^2)^2
    \end{align*}
\item
    \begin{align*}
    &(x + iy + jz + kw)(x + iy + jw - kz) \\
    &= (x^2 - y^2 ) + i(2xy - z^2 - w^2) \\
        &+ j(xz + yz + xw + yw) + k(-xz - yz + xw + yw) 
    \end{align*}
Taking norms, this gives the identity:
    \begin{align*}
    &(x^2 - y^2 )^2 + (2xy - z^2 - w^2)^2 \\
        &+ (xz + yz + xw + yw)^2 + (-xz - yz + xw + yw)^2 \\ 
    &= (x^2 + y^2 + z^2 + w^2)^2
    \end{align*}
\item
    \begin{align*}
    &(x + iy + jz + kw)(x - iy + jw - kz) \\
    &= (x^2 + y^2 ) + i(- z^2 - w^2) \\
        &+ j(xz + yz + xw - yw) + k(-xz + yz + xw + yw) 
    \end{align*}
Taking norms, this gives the identity:
    \begin{align*}
    &(x^2 + y^2 )^2 + (- z^2 - w^2)^2 \\
        &+ (xz + yz + xw - yw)^2 + (-xz + yz + xw + yw)^2 \\
    &= (x^2 + y^2 + z^2 + w^2)^2
    \end{align*}
\item
    \begin{align*}
    &(x + iy + jz + kw)(x + iz + jw + ky) \\
    &= (x^2 - yz - yw - zw ) + i(xy + xz + yz - w^2) \\
        &+ j(-y^2 + xz + xw + zw) + k(xy - z^2 + xw + yw) 
    \end{align*}
Taking norms, this gives the identity:
    \begin{align*}
    &(x^2 - yz - yw - zw )^2 + (xy + xz + yz - w^2)^2 \\
        &+ (-y^2 + xz + xw + zw)^2 + (xy - z^2 + xw + yw)^2 \\
    &= (x^2 + y^2 + z^2 + w^2)^2
    \end{align*}
\item
    \begin{align*}
    &(x + iy + jz + kw)(x + iz + jw - ky) \\
    &= (x^2 - yz + yw - zw ) + i(xy + xz - yz - w^2) \\
        &+ j(y^2 + xz + xw + zw) + k(xw - xy - z^2  + yw) 
    \end{align*}
Taking norms, this gives the identity:
    \begin{align*}
    &(x^2 - yz + yw - zw )^2 + (xy + xz - yz - w^2)^2 \\
        &+ (y^2 + xz + xw + zw)^2 + (xw - xy - z^2  + yw)^2 \\
    &= (x^2 + y^2 + z^2 + w^2)^2
    \end{align*}
\end{enumerate}





\newpage




\newcommand{\journal}[6]{{\sc #1,} #2, {\it #3} {\bf #4} (#5), #6.}
\newcommand{\journalfive}[5]{{\sc #1,} #2, {\it #3}  (#4), #5.}
\newcommand{\preprint}[3]{{\sc #1,} #2, preprint #3.}
\newcommand{\book}[4]{{\sc #1,} #2, #3, #4.}
\newcommand{\collection}[6]{{\sc #1,}  #2, #3, in {\it #4}, #5, #6.}
\newcommand{\JCTA}{J.\ Combin.\ Theory Ser.\ A}
\newcommand{\arxiv}[3]{{\sc #1,} #2, {\tt #3}.}
\newcommand{\article}[3]{{\sc #1,} #2, {\tt #3}.}




\begin{thebibliography}{1}

%\bibitem{Hardy_and_Wright}
%\book{G.\ H.\ Hardy and E.\ M.\ Wright}
%         {An Introduction to the Theory of Numbers, 6th Edition}
%         {Oxford University Press, Oxford} 
%         {2008}



\bibitem{Davidoff_Sarnak_Valette}
\book{G.\ Davidoff, P.\ Sarnak, and A.\ Valette}
         {Elementary Number Theory,
           Group Theory,
           and Ramanujan Graphs}
         {Cambridge University Press}
         {2003}

\bibitem{Dickson}
\book{L.\ E.\ Dickson}
         {History of the Theory of Numbers}
         {AMS Chelsea Publishing}
         {1999}


%\bibitem{Jacobi}
%\journal{C.\ G.\ J.\ Jacobi}
%        {De compositione numerorum e quatuor quadratis}
%        {Journal f\"ur die reine und angewandte Mathematik}
%        {12}{1834}{167--172}


\bibitem{Ferrari}
\journal{F.\ Ferrari}
        {Risoluzione Dell'Equazione}
        {Supplemento al Periodico di Matematica}
        {11}{1908}{129--131}



\bibitem{Mordell}
\book{L.\ J.\ Mordell}
         {Diophantine Equations}
         {Academic Press, London and New York} 
         {1969}


\end{thebibliography}








}%ends bold font

}%ends large font
\end{landscape}
\end{document}
